% Liste des hypothèses à aborder :
%   - Distance entre deux agents est celle à la surface d'un cercle
%     . donner la formule
%     . montrer la page où on parle de ça
%   - Providers disponibles pour un consumer
%     . montrer que c'est l'environnement qui le donne, mais on sait pas comment
%     . citer la phrase "range opérationelle"
%     . donc on renvoie b tq d(a,b) < r_o
%   - Exploration de Boltzmann : Température
%     . pas de valeur donnée
%     . calcul empirique + intuitions
%     . on prend donc T = 1 / ln
%     . simplification à qqch de croissant
%   - Distribution initiale des agents
%     . randomly = uniforme
%     . choisir une méthode uniforme
%     . le trivial est concentré autour des poles.
%   - Performance moyenne des providers pour un agent
%     . utilisé dans l'exploration de Boltzmann
%     . pas précisé comment calculé
%     . approche : moyenne sur toute la vie vs moyenne sur l'historique
%     . les deux se défendent, mais quand profit monte,
%       moyenne monterait aussi. Pas logique ?
%   - Initialisation du μ de chaque provider
%     . pas explicitement dit
%     . table x parle d'une range pour μ donc on peut supposer loi uniforme
%     . sinon on aurait une valeur simple et pas une range
%   - Méthode pour noter une utilité
%     . on peut supposer qu'on map [-10,10] dans [-1,1]
%     . mais pertinent d'imaginer autre chose (extension ?)
%     . du coup, comment on fait l'inverse pour l'exploration Boltzmann (citer ligne & page)
%   - Remplacement départs d'agents
%     . explicitement censés être distribués : donc pas un remplacement exact
%   - Changement de profil et changement de performance simultanés
%     . pas mentionné, mais on propose que l'un ou l'autre

% ================================================================================
\



\begin{hypothese}{Valeur accordée à une utilitée}
\label{hyp:rating}

Dans le modèle, après toute intéraction $i$ entre $a$ et $b$, l'agent $a$ crée un rating $r$ pour chacun des critères $c$ évaluables dans cette interaction. Ce rating prend la forme d'un tuple $r = (a, b, c, i, v)$ où $v \in [-1,1]$ est la valeur que $a$ donne à $b$ pour le critère $c$ lors de l'interaction $i$.\\
Cependant, il n'est nulle part précisé comment l'agent $a$ détermine cette valeur. Nous savons que dans l'implémentation de l'article, $a$ reçoit une utilité \UG $\in [-10,10]$. Nous ferons donc les hypothèses suivantes :
\begin{itemize}
    \item On peut exprimer $v$ comme une fonction de l'\UG\ obtenue par $a$ lors de cette interaction, et cette fonction est propre à $a$ :
    $$ v = \varphi_a(\text{\UG}) $$
    \item Au vu du manque d'informations, on fera l'hypothèse que cette fonction normaliser \UG\ dans l'intervalle de $v$, et qu'elle est la même pour chaque agent :
    $$ \varphi_a(x) = \frac{x}{10}, \forall a $$
\end{itemize}

Cette hypothèse apporte même une autre opportunité d'amélioration au modèle : tous les agents n'ont pas une appréciation linéaire de l'utilité. De même, cette appréciation n'est pas toujours objective. Dans des situations réalistes, certaines personnes notent trop sévèrement, alors que d'autres peuvent être trop indulgentes. Il serait donc intéressant de doter les agents d'utilités non-linéaire dans un premier temps, et peut être par la suite de différencier certains profils d'agents pour intégrer ces disparités dans les notes des différents critères.
\end{hypothese}

\begin{hypothese}{Dégradation du service liée à la distance}

L'article évoque une dégradation linéaire du service lorsque la distance entre les deux agents en interaction, $a$ et $b$, dépasse la portée opérationelle $r_o(b)$ de l'agent fournissant le service.\newline

Cependant, les détails de cette dégradation linéaire ne sont pas donnés par l'article. Cette dégradation est donc une fonction de $x = d(a, b)$ la distance entre $a$ et $b$ qui modifie l'utilité $u$ fournie par l'interaction. On peut tout d'abord supposer :

$$ f_{UG}(x)= 
\begin{cases}
    UG,                 & \text{si } x \leq r_o(b)\\
    UG + \alpha * (x - r_o),  & \text{sinon}
\end{cases} $$
Avec $alpha < 0$ pour obtenir une dégradation.\\

Cependant, il reste quelque problèmes : avec une expression aussi générale, on peut obtenir $u < UG_{min} = -10$, ce qui n'est pas possible. On doit donc introduire une distance maximale à partir de laquelle le service obtenu est minimal, $r_{max}$. On ajoute donc la condition $f_{UG}(x) = UG_{min}, \forall x \geq r_{max}$. Ayant deux points de cette fonction linéaire, on peut alors déterminer que $\alpha = \dfrac{UG_{min} - UG}{r_{max} - r_o}$. Ce qui nous donne comme expression finale :

$$ f_{UG}(x)= 
\begin{cases}
    UG,                 & \text{si } x \leq r_o(b)\\
    UG + \alpha * (x - r_o),  & \text{si } r_o \leq x \leq r_{max}\\
    UG_{min},  & \text{si } x \geq r_{max}
\end{cases} $$

Ou alors, en une seule expression, 

$$ f_{UG}(x) = \max(UG_{min}, \min(UG, UG + \dfrac{UG_{min} - UG}{r_{max} - r_o} * (x - r_0))) $$

On peut vérifier avec un graphe que ce comportement correspond à ce qui est attendu, avec : $UG = 6$, $UG_{min} = -10$, $r_o = 10$, $r_{max} = 50$\newline

% @TODO: mettre les explications des axes.
% @TODO: expliquer la version purement linéaire avec la range qui varie
% @TODO: expliquer que ça sert à rien tout ça

\begin{center}
    \begin{tikzpicture}
        \pgfmathdeclarefunction{f}{4}{% UG, UGmin, ro, rmax
          \pgfmathparse{max(#2, min(#1, #1 + ((#2 - #1) / ( #4 - #3 )) * (x - #3)))}%
        }
        \begin{axis}[every axis plot post/.append style={
          mark=none,domain=0:60,samples=50,smooth},
          axis x line*=center, % no box around the plot, only x and y axis
          axis y line*=left, % the * suppresses the arrow tips
          enlarge x limits=false,
          enlarge y limits=true]
          \addplot [blue, thick] {f(6, -10, 10, 50)};
        \end{axis}
    \end{tikzpicture}
\end{center}

Cette interprétation nous semble la plus logique. Cependant, il faut choisir une valeur pour $r_{max}$. Comme ce choix est arbitraire, nous avons choisi $r_{max} = 2 * r_o$.\newline

% @TODO: préciser que c'est pas nécessairement utile, au vu des cas particuliers qu'on s'impose dans l'implémentation

\end{hypothese}

\begin{hypothese}{Répartition des agents dans l'espace}
L'article ne précise pas la loi de probabilité suivie par les coordonnée des agents. C'est un détail important au vu de l'importance de la topologie dans cette simulation. Nous partons du principe que la distribution souhaitée par les auteurs est une distribution uniforme sur l'ensemble de la sphère. Il existe plusieurs méthode pour obtenir une telle distribution.\newline

La première est de considérer les coordonnées sphériques $r, \varphi, \theta$, avec $r \in [0, 1]$, $\varphi \in [0, \pi[$, $\theta \in ]-\pi, +\pi]$. On définit alors la position d'un nouvel agent comme le tirage aléatoire de trois variables uniformes $r, \varphi, \theta$ sur leurs espaces de définitions respectifs.\newline

La seconde consiste à tirer des points uniformes dans un cube en coordonnées cartésiennes : $(x, y, z) \in [-1, 1]^3$, puis rejeter les points qui sont en dehors de la sphère unitaire ($\|(x, y, z)\| > 1$).\newline

Nous avons choisi la première méthode, car elle est plus efficace tout en donnant des résultats corrects.
\end{hypothese}
\vspace{10pt}
