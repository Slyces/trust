% Liste des hypothèses à aborder :
%   - Distance entre deux agents est celle à la surface d'un cercle
%     . donner la formule
%     . montrer la page où on parle de ça
%   - Providers disponibles pour un consumer
%     . montrer que c'est l'environnement qui le donne, mais on sait pas comment
%     . citer la phrase "range opérationelle"
%     . donc on renvoie b tq d(a,b) < r_o
%   - Exploration de Boltzmann : Température
%     . pas de valeur donnée
%     . calcul empirique + intuitions
%     . on prend donc T = 1 / ln
%     . simplification à qqch de croissant
%   - Distribution initiale des agents
%     . randomly = uniforme
%     . choisir une méthode uniforme
%     . le trivial est concentré autour des poles.
%   - Performance moyenne des providers pour un agent
%     . utilisé dans l'exploration de Boltzmann
%     . pas précisé comment calculé
%     . approche : moyenne sur toute la vie vs moyenne sur l'historique
%     . les deux se défendent, mais quand profit monte,
%       moyenne monterait aussi. Pas logique ?
%   - Initialisation du μ de chaque provider
%     . pas explicitement dit
%     . table x parle d'une range pour μ donc on peut supposer loi uniforme
%     . sinon on aurait une valeur simple et pas une range
%   - Méthode pour noter une utilité
%     . on peut supposer qu'on map [-10,10] dans [-1,1]
%     . mais pertinent d'imaginer autre chose (extension ?)
%     . du coup, comment on fait l'inverse pour l'exploration Boltzmann (citer ligne & page)
%   - Remplacement départs d'agents
%     . explicitement censés être distribués : donc pas un remplacement exact
%   - Changement de profil et changement de performance simultanés
%     . pas mentionné, mais on propose que l'un ou l'autre

% ================================================================================
\



\begin{hypothese}{Valeur accordée à une utilitée}
\label{hyp:rating}

Dans le modèle, après toute interaction $i$ entre $a$ et $b$, l'agent $a$ crée un rating $r$ pour chacun des critères $c$ évaluables dans cette interaction. Ce rating prend la forme d'un tuple $r = (a, b, c, i, v)$ où $v \in [-1,1]$ est la valeur que $a$ donne à $b$ pour le critère $c$ lors de l'interaction $i$.\newline
Cependant, il n'est nulle part précisé comment l'agent $a$ détermine cette valeur. Nous savons que dans l'implémentation de l'article, $a$ reçoit une utilité \UG $\in [-10,10]$. Nous ferons donc les hypothèses suivantes :

\begin{itemize}
    \item On peut exprimer $v$ comme une fonction de l'\UG\ obtenue par $a$ lors de cette interaction, et cette fonction est propre à $a$ :
    $$ v = f_a(\text{\UG}) $$
    \item Au vu du manque d'informations, on fera l'hypothèse que cette fonction normalise \UG\ dans l'intervalle de $v$, et qu'elle est la même pour chaque agent :
    $$ f_a(x) = \frac{x}{10}, \forall a $$
\end{itemize}

Cette hypothèse est triviale, mais pourrait représenter une opportunité d'améliorer le modèle : tous les agents n'ont pas une appréciation linéaire de l'utilité. De même, cette appréciation n'est pas toujours objective. Dans des situations réalistes, certaines personnes notent trop sévèrement, alors que d'autres peuvent être trop indulgentes. Il serait donc intéressant de doter les agents d'utilités non-linéaire dans un premier temps, et peut être par la suite de différencier certains profils d'agents pour intégrer ces disparités dans les notes des différents critères.\newline

Cette amélioration (assez simple à mettre en place) correspond à des comportements réels observés (la subjectivité) et peut venir se placer en complément / extension des comportements de mensonges, qui sont difficiles à distinguer d'un point de vue extérieur.

\end{hypothese}

\begin{hypothese}{Dégradation du service liée à la distance}

L'article évoque une dégradation linéaire du service lorsque la distance entre les deux agents en interaction, $a$ et $b$, dépasse la portée opérationelle $r_o(b)$ de l'agent fournissant le service. Cependant, les détails de cette dégradation linéaire ne sont pas donnés par l'article. Nous allons décrire ci-dessous notre réflexion pour traiter cette hypothèse. \newline

L'utilité perçue est donc une fonction de $x = d(a, b)$ la distance entre $a$ et $b$. Dans un premier temps, on sait que cette fonction sera constante sur $0 \leq x \leq r_o$.

$$ f_{UG}(x)= 
\begin{cases}
    UG,                 & \text{si } x \leq r_o(b)\\
    UG + \alpha * (x - r_o),  & \text{sinon}
\end{cases} $$
Avec $alpha < 0$ pour obtenir une dégradation.\newline

Cependant, il reste quelque problèmes : avec une expression aussi générale, on peut obtenir $u < UG_{min} = -10$, ce qui n'est pas censé être possible. Ici, deux options s'offrent à nous :
\begin{itemize}
    \item Etablir un coefficient constant $\alpha$ de dégradation, peu importe l'utilité émise par l'agent. Ce qui veut dire que la portée maximale $r_{max}$ où $u \leq -10$ varie d'une intéraction à l'autre.
    \item Etablie une portée maximale fixe, $r_{max}$, et s'arranger pour que chaque agent situé entre $r_o$ et $r_{max}$ reçoive une utilité $u \leq -10$, pour toute interaction.
\end{itemize}

La première option est intéressante, mais pose le problème du choix du coefficient. Il est arbitraire mais assez peu de choix logiques semblent s'imposer. La seconde option permet de simplement fixer $r_{max}$, potentielement en fonction de $r_o$, et simplifie alors le paramétrage. Nous avons choisi la deuxième, mais nous détaillerons en fin de section l'impact final attendu de cette partie du modèle.

On ajoute donc la condition $f_{UG}(x) = UG_{min}, \forall x \geq r_{max}$. Ayant deux points de cette fonction linéaire, on peut alors déterminer que $\alpha = \dfrac{UG_{min} - UG}{r_{max} - r_o}$. Ce qui nous donne comme expression finale :

$$ f_{UG}(x)= 
\begin{cases}
    UG,                 & \text{si } x \leq r_o(b)\\
    UG + \alpha * (x - r_o),  & \text{si } r_o \leq x \leq r_{max}\\
    UG_{min},  & \text{si } x \geq r_{max}
\end{cases} $$

Ou alors, en une seule expression, 

$$ f_{UG}(x) = \max(UG_{min}, \min(UG, UG + \dfrac{UG_{min} - UG}{r_{max} - r_o} * (x - r_0))) $$

On peut vérifier avec un graphe que ce comportement correspond à ce qui est attendu, avec : $UG = 6$, $UG_{min} = -10$, $r_o = 10$, $r_{max} = 50$\newline

% @TODO: mettre les explications des axes.
% @TODO: expliquer la version purement linéaire avec la range qui varie
% @TODO: expliquer que ça sert à rien tout ça

\begin{center}
    \begin{tikzpicture}
        \pgfmathdeclarefunction{f}{4}{% UG, UGmin, ro, rmax
          \pgfmathparse{max(#2, min(#1, #1 + ((#2 - #1) / ( #4 - #3 )) * (x - #3)))}%
        }
        \begin{axis}[every axis plot post/.append style={
          mark=none,domain=0:60,samples=50,smooth},
          axis x line*=center, % no box around the plot, only x and y axis
          axis y line*=left, % the * suppresses the arrow tips
          enlarge x limits=false,
          enlarge y limits=true]
          \addplot [blue, thick] {f(6, -10, 10, 50)};
        \end{axis}
    \end{tikzpicture}
\end{center}

\noindent Cette méthode revient donc à dire que si $a$ est à mi-distance entre $r_o$ et $r_{max}$, $a$ recevra toujours $\frac{u - 10}{2}$, au lieu de recevoir $u - K$ avec $K$ constant (à distance fixe).\newline

Le choix de $r_{max}$ est arbitraire, nous avons donc introduit un coefficient $y \req 0$ qui nous donne $r_{max} = r_o * (1 + y)$.\newline

% /!\
\greysep

Pour déterminer si oui, ou non, les résultats de l'article utilisent une forme de réduction linéaire de la distance, nous nous somme intéressés à la figure 5. Celle-ci montre les performances d'un agent aléatoire, située autour de $-1$. Lorsqu'on simule un monde où chaque agent ne voit que les providers à sa portée, l'utilitée moyenne d'un agent aléatoire est de $1.5$ (en utilisant les proportions de providers fournies). La figure 5 utilise donc une réduction des performances des providers. Et effectivement, nous arrivons à reproduire un agent aléatoire à $-1$ en faisant varier $r_{max}$.\newline

Ci-dessous, une comparaison des performances avec et sans réduction de la distance (sans réduction : un agent $a$ ne peut contacter un provider $b$ que si $d(a,b) < b.r_o$).

%\includegraphics[]{images/}

% @TODO: préciser que c'est pas nécessairement utile, au vu des cas particuliers qu'on s'impose dans l'implémentation

\end{hypothese}

\vspace{10pt}
