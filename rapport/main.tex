%! TEX program = lualatex
\documentclass{article}
% -------------------------------------------------------------
%\usepackage[utf8]{inputenc}
 \usepackage{fontspec}
%\usepackage[T1]{fontenc}
\usepackage[francais]{babel}
\usepackage{style}

% ------------------------------------------------------------
% Gestion des marges
\usepackage[top=2.5cm, bottom=2.5cm, left=2cm, right=2cm]{geometry}

% -------------------------------------------------------------
% Page de garde
\title{%
    \vspace*{\fill}
    \textbf{\scshape \tcol{Mo}délisation et \tcol{Si}mulation \tcol{M}ulti\tcol{A}gents}\\
    Rapport de Projet
}
\author{
    Maxime Desbois et Simon Lassourreuille
}
\date{
    Décembre 2018
    \vspace*{\fill}
}

\begin{document}

\maketitle

\clearpage

% ================================================================================ %
\section{Hypothèses d'implémentation}%
\label{sec:hypotheses_d_implementation}

% Liste des hypothèses à aborder :
%   - Distance entre deux agents est celle à la surface d'un cercle
%     . donner la formule
%     . montrer la page où on parle de ça
%   - Providers disponibles pour un consumer
%     . montrer que c'est l'environnement qui le donne, mais on sait pas comment
%     . citer la phrase "range opérationelle"
%     . donc on renvoie b tq d(a,b) < r_o
%   - Exploration de Boltzmann : Température
%     . pas de valeur donnée
%     . calcul empirique + intuitions
%     . on prend donc T = 1 / ln
%     . simplification à qqch de croissant
%   - Distribution initiale des agents
%     . randomly = uniforme
%     . choisir une méthode uniforme
%     . le trivial est concentré autour des poles.
%   - Performance moyenne des providers pour un agent
%     . utilisé dans l'exploration de Boltzmann
%     . pas précisé comment calculé
%     . approche : moyenne sur toute la vie vs moyenne sur l'historique
%     . les deux se défendent, mais quand profit monte,
%       moyenne monterait aussi. Pas logique ?
%   - Initialisation du μ de chaque provider
%     . pas explicitement dit
%     . table x parle d'une range pour μ donc on peut supposer loi uniforme
%     . sinon on aurait une valeur simple et pas une range
%   - Méthode pour noter une utilité
%     . on peut supposer qu'on map [-10,10] dans [-1,1]
%     . mais pertinent d'imaginer autre chose (extension ?)
%     . du coup, comment on fait l'inverse pour l'exploration Boltzmann (citer ligne & page)
%   - Remplacement départs d'agents
%     . explicitement censés être distribués : donc pas un remplacement exact
%   - Changement de profil et changement de performance simultanés
%     . pas mentionné, mais on propose que l'un ou l'autre

% ================================================================================
\



\begin{hypothese}{Valeur accordée à une utilitée}
\label{hyp:rating}

Dans le modèle, après toute intéraction $i$ entre $a$ et $b$, l'agent $a$ crée un rating $r$ pour chacun des critères $c$ évaluables dans cette interaction. Ce rating prend la forme d'un tuple $r = (a, b, c, i, v)$ où $v \in [-1,1]$ est la valeur que $a$ donne à $b$ pour le critère $c$ lors de l'interaction $i$.\\
Cependant, il n'est nulle part précisé comment l'agent $a$ détermine cette valeur. Nous savons que dans l'implémentation de l'article, $a$ reçoit une utilité \UG $\in [-10,10]$. Nous ferons donc les hypothèses suivantes :
\begin{itemize}
    \item On peut exprimer $v$ comme une fonction de l'\UG\ obtenue par $a$ lors de cette interaction, et cette fonction est propre à $a$ :
    $$ v = \varphi_a(\text{\UG}) $$
    \item Au vu du manque d'informations, on fera l'hypothèse que cette fonction normaliser \UG\ dans l'intervalle de $v$, et qu'elle est la même pour chaque agent :
    $$ \varphi_a(x) = \frac{x}{10}, \forall a $$
\end{itemize}

Cette hypothèse apporte même une autre opportunité d'amélioration au modèle : tous les agents n'ont pas une appréciation linéaire de l'utilité. De même, cette appréciation n'est pas toujours objective. Dans des situations réalistes, certaines personnes notent trop sévèrement, alors que d'autres peuvent être trop indulgentes. Il serait donc intéressant de doter les agents d'utilités non-linéaire dans un premier temps, et peut être par la suite de différencier certains profils d'agents pour intégrer ces disparités dans les notes des différents critères.
\end{hypothese}

\begin{hypothese}{Dégradation du service liée à la distance}

L'article évoque une dégradation linéaire du service lorsque la distance entre les deux agents en interaction, $a$ et $b$, dépasse la portée opérationelle $r_o(b)$ de l'agent fournissant le service.\newline

Cependant, les détails de cette dégradation linéaire ne sont pas donnés par l'article. Cette dégradation est donc une fonction de $x = d(a, b)$ la distance entre $a$ et $b$ qui modifie l'utilité $u$ fournie par l'interaction. On peut tout d'abord supposer :

$$ f_{UG}(x)= 
\begin{cases}
    UG,                 & \text{si } x \leq r_o(b)\\
    UG + \alpha * (x - r_o),  & \text{sinon}
\end{cases} $$
Avec $alpha < 0$ pour obtenir une dégradation.\\

Cependant, il reste quelque problèmes : avec une expression aussi générale, on peut obtenir $u < UG_{min} = -10$, ce qui n'est pas possible. On doit donc introduire une distance maximale à partir de laquelle le service obtenu est minimal, $r_{max}$. On ajoute donc la condition $f_{UG}(x) = UG_{min}, \forall x \geq r_{max}$. Ayant deux points de cette fonction linéaire, on peut alors déterminer que $\alpha = \dfrac{UG_{min} - UG}{r_{max} - r_o}$. Ce qui nous donne comme expression finale :

$$ f_{UG}(x)= 
\begin{cases}
    UG,                 & \text{si } x \leq r_o(b)\\
    UG + \alpha * (x - r_o),  & \text{si } r_o \leq x \leq r_{max}\\
    UG_{min},  & \text{si } x \geq r_{max}
\end{cases} $$

Ou alors, en une seule expression, 

$$ f_{UG}(x) = \max(UG_{min}, \min(UG, UG + \dfrac{UG_{min} - UG}{r_{max} - r_o} * (x - r_0))) $$

On peut vérifier avec un graphe que ce comportement correspond à ce qui est attendu, avec : $UG = 6$, $UG_{min} = -10$, $r_o = 10$, $r_{max} = 50$\newline

% @TODO: mettre les explications des axes.
% @TODO: expliquer la version purement linéaire avec la range qui varie
% @TODO: expliquer que ça sert à rien tout ça

\begin{center}
    \begin{tikzpicture}
        \pgfmathdeclarefunction{f}{4}{% UG, UGmin, ro, rmax
          \pgfmathparse{max(#2, min(#1, #1 + ((#2 - #1) / ( #4 - #3 )) * (x - #3)))}%
        }
        \begin{axis}[every axis plot post/.append style={
          mark=none,domain=0:60,samples=50,smooth},
          axis x line*=center, % no box around the plot, only x and y axis
          axis y line*=left, % the * suppresses the arrow tips
          enlarge x limits=false,
          enlarge y limits=true]
          \addplot [blue, thick] {f(6, -10, 10, 50)};
        \end{axis}
    \end{tikzpicture}
\end{center}

Cette interprétation nous semble la plus logique. Cependant, il faut choisir une valeur pour $r_{max}$. Comme ce choix est arbitraire, nous avons choisi $r_{max} = 2 * r_o$.\newline

% @TODO: préciser que c'est pas nécessairement utile, au vu des cas particuliers qu'on s'impose dans l'implémentation

\end{hypothese}

\begin{hypothese}{Répartition des agents dans l'espace}
L'article ne précise pas la loi de probabilité suivie par les coordonnée des agents. C'est un détail important au vu de l'importance de la topologie dans cette simulation. Nous partons du principe que la distribution souhaitée par les auteurs est une distribution uniforme sur l'ensemble de la sphère. Il existe plusieurs méthode pour obtenir une telle distribution.\newline

La première est de considérer les coordonnées sphériques $r, \varphi, \theta$, avec $r \in [0, 1]$, $\varphi \in [0, \pi[$, $\theta \in ]-\pi, +\pi]$. On définit alors la position d'un nouvel agent comme le tirage aléatoire de trois variables uniformes $r, \varphi, \theta$ sur leurs espaces de définitions respectifs.\newline

La seconde consiste à tirer des points uniformes dans un cube en coordonnées cartésiennes : $(x, y, z) \in [-1, 1]^3$, puis rejeter les points qui sont en dehors de la sphère unitaire ($\|(x, y, z)\| > 1$).\newline

Nous avons choisi la première méthode, car elle est plus efficace tout en donnant des résultats corrects.
\end{hypothese}
\vspace{10pt}


% ================================================================================ %
\section{Pseudocode}

\begin{pseudocode}{Initialisation : Consumer}
\begin{minted}{latex}
pour chaque consumer a faire
    initialiser une liste vide de taille H
    initialiser r_o
    choisir un niveau d'activité alpha ~ U([0.25, 1.00])
    choisir une position au hasard
fin pour
\end{minted}
\end{pseudocode}

\begin{pseudocode}{Initialisation : Producer}
\begin{minted}{latex}
pour chaque provider b faire
    déterminer le type de b parmi {good, ok, bad, intermitent}
    initialiser r_o
    si b n'est pas intermitent alors
        initialiser σp
        choisir au hasard μp dans l'intervale correspondant
    fin si
fin pour
\end{minted}
\end{pseudocode}

\begin{pseudocode}{Déroulement d'un round}
\begin{minted}{latex}
pour chaque consumer a faire
    déterminer si a nécessite un service
    si vrai alors
        incrémenter i le compteur d'intéractions
        b <-- choix d'un provider par a
        ug <-- b génère au hasard une utilité entre -10 et 10
        v <-- déterminer la valeur de l'intéraction
        créer un rating r = (a, b, c, i, v)
    fin si
fin pour

# simulation d'instantanéité
pour chaque consumer a faire
    si r existe alors
        enregistrer le rating r dans la liste de a
        envoyer le rating r à b
        b enregistre r si v est meilleur que son moins bon référencement
    fin si
fin pour
\end{minted}
\end{pseudocode}

\begin{pseudocode}{Choix d'un provider}
\begin{minted}{latex}
P <-- récupérer les providers disponibles
pour chaque b dans P faire
    v <-- déterminer la valeur de confiance de b
    si v est défini alors
        insérer b dans la liste NoTrust sinon
        insérer b dans la liste Trust
    fin si
fin pour
trier la liste Trust par valeur de confiance
choisir stratégie (exploration ou exploitation)
si exploration alors
    p <-- choisir un provider au hasard dans NoTrust
sinon
    p <-- provider avec la plus grande valeur de confiance
fin si
retourner p
\end{minted}
\end{pseudocode}

\begin{pseudocode}{Déterminer la valeur de confiance de la composante K}
\begin{minted}{latex}
Entrée: RK ensemble de ratings pour le calcul de la composante K

Σω <-- 0
Σω_vi <-- 0
pour chaque ri dans RK faire
    ω <-- calculer ω_K(ri)
    Σω <-- Σω + ω
    Σω_vi <-- Σω + ω * vi
fin pour
T_K <-- Σω_vi / Σω
ρ_RK <-- 1 - exp(-γK * Σω)
Σω_vi_T <-- 0
pour chaque ri dans RK faire
    Σω_vi_T <-- Σω_vi_T + ω_K(ri) * abs(vi - T_K)
fin pour
ρ_DK = 1 - 0.5 * (Σω_vi_T / Σω)
ρ_K = ρ_RK * ρ_DK

retourner (ρ_K, T_K)
\end{minted}
\end{pseudocode}

\begin{pseudocode}{Déterminer la valeur de confiance générale}
\begin{minted}{latex}
# Déterminer RI l'ensemble des ratings d'Intéraction
\end{minted}
\end{pseudocode}

% Idée d'optimisation
%    Quand l'action choisir est explorer, arrêter le calcul
%    de confiance dès qu'on sait que la valeur est définie


\section{Etude préliminaire}
\label{sec:etude-preliminaire}

\question{Résumez en quelque ligne la problématique des auteurs, et les choix de modélisation qu'ils proposent.}

Les auteurs proposent un modèle de confiance permettant à un agent de juger de la qualité des services d'autres agents. Plus précisément, ce modèle doit être décentralisé, robuste aux mensonges.

\question{Pourquoi ? Test.}


% ================================================================================ %
\section{Etudes progressive du modèle FIRE}
\label{sec:etude-progressive}

\smalltitle{Conception de l'interface NetLogo}

Est-ce qu'on en parle ? Si oui, c'est ici !

% ================================================================================ %
\section{Modèle avec composante IT seule (FIRE-IT)}

\question{Écrivez en pseudo-code l'intégralité des algorithmes que vous allez implémenter pour simuler le modèle FIRE-IT. Expliquez et Justifiez vos choix (notamment pour l'implémentation de la topologie sphérique en Netlogo).}

\smalltitle{}

\question{Décrivez le protocole expérimental (et votre procédure d'initialisation des agents en particulier), et donnez dans un tableau la valeur de tous les paramètres que vous fixez pour reproduire la Figure 9.}

\smalltitle{Initialisation des Agents}

Comme discuté dans la section 

\question{Affichez votre version de la Figure 9 et commentez les résultats.}

% ================================================================================ %
\section{Modèle avec composante IT et WR (FIRE-IT-WR)}

\question{Écrivez en pseudo-code l'intégralité des algorithmes que vous allez implémenter pour simuler le modèle FIRE-IT-WR. Expliquez et Justifiez vos choix. Implémentez en Netlogo.}

\question{Indiquez le protocole pour simuler la Figure, et les éventuels changements dans les paramètres (modifications des anciennes valeurs, éventuels ajouts de nouveaux paramètres)}

\question{Affichez votre version de la figure 9 complète et expliquez les résultats.}

\section{Modèle avec composante IT, WR et CR (FIRE-IT-WR-CR)}

\question{Écrivez en pseudo-code l'intégralité des algorithmes que vous allez implémenter l'ajout de la composante CR. Expliquez et Justifiez vos choix. Implémentez en Netlogo.}

\question{Indiquez le protocole pour simuler ces Figures, et les éventuels changements dans les paramètres (modifications des anciennes valeurs, éventuels ajouts de nouveaux paramètres).}

\question{Affichez vos version des Figures 10 et 11 et expliquez les résultats, notamment en comparaison de la Figure 9.}

\section{Analyses de sensibilité du modèle FIRE-IT-WR-CR.}

\question{Affichez un graphique 3D (avec un tableur ou grapheur) qui prend comme axes x et y les poids Wc et Ww et en axe z la performance moyenne des UG. Expliquez comment vous avez obtenu ce graphique et commentez les résultats.}

\question{Affichez maintenant (avec les paramètres initiaux de [1]) un histogramme montrant la distribution des valeurs des UGs des clients. Commentez.}

\question{Faites varier individuellement les paramètres NPG, NPO et NB=NPI+NPB (mais en respectant toujours NPG+NPO+NB=100). Dans des graphes séparés, afficher la moyenne et l'écart type des UG pour chacune 10 ces valeurs. Commentez. Quel est effet de ces paramètres sur la distribution des UG des clients ?}

\section{Extension -- }

\section{Bonus}

\question{Expliquez en quelques lignes comment le modèle FIRE pourrait être adapté et étendu pour simuler une société telle qu'elle est décrite dans l'épisode Nosedive de la série Black Mirror.}

\end{document}
